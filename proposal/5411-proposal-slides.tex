\documentclass{beamer}
\usepackage[utf8]{inputenc}
\usepackage{utopia}
\usepackage[round]{natbib}
%\usepackage[
%backend=biber,
%style=plainnat,
%citestyle=authoryear
%]{biblatex}

\bibliographystyle{plainnat}

\usetheme{Madrid}
\usecolortheme{default}
\title[COMP5411 Project Proposal]{COMP5411 Rendering Project Proposal \\ Magnifying Lenses}
\author[Anshuman \& Aaron]{Anshuman Medhi \\ Aaron Si-yuan Wang}
\date{Group 21}

\begin{document}
\frame{\titlepage}

\begin{frame}
    \frametitle{Project Description}
    \begin{block}{Rendering Lenses}
       \begin{itemize}
        \item Render a standard scene with objects and lighting, and include interactive magnifying lenses
        \item Simulate and render all the kinds of distortion associated with lenses
       \end{itemize} 
    \end{block}
\begin{figure}[htpb]
    \centering
        \includegraphics[width=0.3\textwidth]{images/mag}
        %\label{fig:}
    \end{figure}
\end{frame}

\begin{frame}
    \frametitle{Project Description}
    \begin{block}{Features}
        \begin{itemize}
        \item Varying the lighting and objects in the scene behind the lenses
        \item Varying parameters of the lens (geometric properties)
        \begin{itemize}
            \item Calculate the optical parameters from the geometric description
        \end{itemize}
        \item Interactively move around the lenses in 3D space
        \item Simulating light dispersion through the lenses (chromatic aberrations)
        \item Simulating various kinds of lens distortions (such as barrel or pincushion distortion)
        \end{itemize}
    \end{block}
    \begin{block}{Challenges}
       \begin{itemize}
           \item Implement realistic approximations as shaders (rather than ray tracing)
           \item Calculate optical properties of lenses from (customizable) geometric description
       \end{itemize} 
    \end{block}
\end{frame}




\end{document}
